\documentclass{article}
\begin{document}
	\pagestyle{empty}
	Thanks for your careful review of my manuscript. According to my understanding, your major
	questions can be divided in the following categories:
	\begin{enumerate}
		\item How the phase transition is defined in this article?
		\item Can the methods used in this manuscript be extended to solve problems of non-equilibrium two-dimensional Ising model?
	\end{enumerate}
	
	For the first question, I study the first order phase transition of the error probability of the estimator $\hat{X}^*$, which is the probability for the Ising model to take values in $S_k(X)$.
	This quantity becomes discontinuous at a critical inverse temperature $\beta=\beta^*$. In you
	comment, you mentioned three cases for 2D Ising model. I think my study is more similar with
	the standard equilibrium statistical mechanics setting when no external magnetic field is present
	and the phase transition occurs by varying the temperature. In the manuscript, the main difference to the classical 2D Ising model lies at the following constructions:
	\begin{itemize}
		\item We define the Ising model on the random graph, instead of fixed 2D lattice;
		\item We add extra terms between nodes without edge connection to the Hamiltonian.
	\end{itemize}
	
	For the second question, I investigated the two articles, which studies the non-equilibrium phase transition of 2D Ising model with external magnetization reservoirs. I think these two articles
	draw their conclusions mainly based on numerical simulation while this manuscript uses probability
	theory to prove the results. I believe theoretical analysis on non-equilibrium two-dimensional Ising model is possible but it may need extra mathematical theory to achieve such purpose.
\end{document}


